\Cref{chap:dynamic_game_theory} showed that equilibria in repeated games and stochastic games are complicated entities.
At best, at equilibrium in a perfect-monitoring repeated game, each agent solves \aac{mdp}.
At worst, in a private-monitoring stochastic game, each solves \aac{pomdp}.
This chapter introduces a new equilibrium concept for stochastic games in which each agent only solves \aac{mdp}.
As was the case in \cref{chap:static_game_theory,chap:dynamic_game_theory}, the concept is first introduced through a single-agent point of view.
Then, the full-fledged multiagent case is exposed.

The following presentation of this research was first developed in~\cite{dudebout_shamma:2012}.

\section{Single-agent Setup}

Consider a discrete-time dynamical system governed by
\begin{equation}
\label{eq:private_dynamic}
x\nxt \drawn  f \of{x, a, s\nxt},
\end{equation}
where~\(x\) is a state, \(a\) is an action, and~\(s\) is a signal.
Variables~\(x\), \(a\), and~\(s\) take values in finite sets~\(\cX\), \(\cA\), and~\(\cS\), respectively.
The agent picks the action~\(a\).
Nature determines the signal~\(s\nxt\) according to
\begin{subequations}
\label{eq:nature_system}
\begin{align}
\label{eq:nature_dynamic}
w\nxt &\drawn n \of{w, x, a}, \\
\label{eq:nature_signal}
s\nxt &\drawn \nu \of{w\nxt},
\end{align}
\end{subequations}
where~\(w\) is a state of Nature evolving in the finite state space~\(\cW\).
The agent observes~\(s\) but not~\(w\).
Denote by~\(\rbN\) the dynamical system described by~\cref{eq:private_dynamic,eq:nature_system}.
Think of this system as a perturbed \ac{mdp}.
The block diagram associated with~\cref{eq:private_dynamic} will be used later in this chapter.
It is represented in \cref{fig:block_diagram_mdp_with_signal_exploded}.

\begin{figure}[htp]
\begin{subfigure}[b]{0.5\textwidth}
\centering
\inputtikz{block_diagram_mdp_with_signal_exploded}
\caption{Full representation}
\end{subfigure}%
\begin{subfigure}[b]{0.5\textwidth}
\centering
\inputtikz{block_diagram_mdp_with_signal}
\caption{Condensed representation}
\end{subfigure}
\ccaption[Block diagram for the \ac{mdp} with perturbation signal.]{
The contour of the block in the condensed representation is doubled.
This serves as a visual reminder that the output variable is fed back into the block.
}
\label{fig:block_diagram_mdp_with_signal_exploded}
\end{figure}

Define the agent's observation by~\(o = \tuple{x, a, s\nxt}\) and the actual realization of the system by~\(r = \tuple{w, x, a, s\nxt}\).
At time~\(t\), the agent's private history is~\(p\Tt = \tuple{\sseqcom{o}{\tm}{0}{1}{t}} \in \cP\Tt\) and the true history is~\(h\Tt = \tuple{\sseqcom{r}{\tm}{0}{1}{t}}\).
Denote by~\(\cP = \union \idxin{t}{\bN} \cP\Tt\) the set of finite private histories.
A strategy~\(\sigma : \cP \to \distribover{\cA}\) is a mapping from private histories to a distribution over the actions.
The agent knowledge of this perturbed \ac{mdp} is represented in \cref{fig:agent_knowledge_single-agent_empirical}, using the diagram format introduced in \cref{chap:dynamic_game_theory}.

\begin{sidewaysfigure}
\centering
\inputtikz{agent_knowledge_single-agent_empirical}
\ccaption[Agent knowledge in the \ac{mdp} with perturbation signal.]{}
\label{fig:agent_knowledge_single-agent_empirical}
\end{sidewaysfigure}

At each time step, the agent receives a payoff according to the utility function~\(u: \cX \times \cA \times \cS \to \bR\).
For a given infinite private history the agent receives the sum of discounted payoffs~\(\sum\idxfromto{t}{0}{\infty} \delta^t u \of{x\Tt, a\Tt, s\Tp}\), where~\(\delta \in \interval[co]{0}{1}\) is a discount factor.
The agent wants to find a strategy maximizing its expected sum of discounted payoffs
\[U\params{\rbN, \sigma}\of{x} = \expectcond[\rbN, \sigma]{\sum\idxfromto{t}{0}{\infty} \delta^t u \of{x\Tt, a\Tt, s\Tp}}{x\Tz = x}.
\]

When the agent knows~\cref{eq:private_dynamic,eq:nature_system}, it is facing a \ac{pomdp}.
To see why, compare~\cref{fig:agent_knowledge_single-agent_empirical} with \cref{fig:agent_knowledge_pomdp}.
The~\ac{pomdp} has state~\(\tuple{w, x}\) and observed value~\(\tuple{x,s}\).
A natural solution concept for this type of problems is an optimal policy for the \ac{pomdp}.
As described in~\cref{sec:pomdps}, the agent computes an optimal policy using beliefs, which are probability distributions over states.
Beliefs are obtained from the private histories~\(p\Tt\), the signaling structure~\cref{eq:nature_system}, and the application of Bayes' rule.
Belief computation is intractable because the size of the belief space grows with time.

The present research is interested in the case where the agent knows~\cref{eq:private_dynamic} but does not know~\cref{eq:nature_system}.
As previously mentioned, this is \aac{mdp} with a perturbation.
The agent fully understands the effect of the perturbation~\(s\) on~\cref{eq:private_dynamic} but does not know how this perturbation is generated.
This block diagram for this setting is shown in \cref{fig:block_diagram_single-agent_problem}.
The information known to the agent is highlighted in~\cref{fig:agent_knowledge_single-agent_empirical_emphasized}.
In such a setting, a less constraining solution concept is required.
Empirical-evidence optimality is one such solution concept that relies on the notion of statistical consistency.

The following section presents the simplest notion of statistical consistency, depth-\(k\) consistency.

\begin{figure}[htp]
\centering
\inputtikz{block_diagram_single-agent_problem}
\ccaption[Block diagram for the single-agent empirical-evidence setup~\(\rbN\).]{
The agent is aware of facing \aac{mdp} with an unknown perturbation signal.
The noisy contour of Nature emphasizes that the agent does not know anything about it.
Furthermore, Nature has access to all the variables to compute~\(s\), but these dependencies are not represented.
}
\label{fig:block_diagram_single-agent_problem}
\end{figure}

\begin{sidewaysfigure}
\centering
\inputtikz{agent_knowledge_single-agent_empirical_emphasized}
\ccaption[Agent knowledge in the single-agent empirical-evidence setup~\(\rbN\).]{
The agent only acknowledges the causality materialized by the bold arrows.
In particular, its signal is underspecified.
From the agent's perspective, the signal arrows have an unknown source.
}
\label{fig:agent_knowledge_single-agent_empirical_emphasized}
\end{sidewaysfigure}

\section{Depth-\texorpdfstring{\(k\)}{k} Consistency}
\label{sec:depth-k_consistency}

The notion of consistency used in this research is best introduced through an example.

\begin{example}[Binary Signal]
Suppose an agent receives a binary signal.
Furthermore, suppose the agent has no information about the underlying generating process.
When the agent observes a realization of this binary signal, it can compute certain parameters.

One of the simplest set of parameters is the probabilities of~\(0\)s and~\(1\)s.
For example, if the agent observes the following sequence:
\[
S\param{\rlA} = 011011011011011011\dots,
\]
it would compute the following parameters:
\[
\probaof[\rlA]{0} = \frac{1}{3} \qquad \text{and}\qquad \probaof[\rlA]{1} = \frac{2}{3}.
\]
Similarly, if the agent observes this other sequence:
\[
S\param{\rlB}= 001111001111001111\dots,
\]
it would compute
\[
\probaof[\rlB]{0} = \frac{1}{3} \qquad \text{and}\qquad \probaof[\rlB]{1} = \frac{2}{3}.
\]
If the probabilities of~\(0\)s and~\(1\)s are the only parameters used by the agent, it would not differentiate~\(S\param{\rlA}\) and~\(S\param{\rlB}\).
Sequences~\(S\param{\rlA}\) and~\(S\param{\rlB}\) are said to be depth-\(0\) consistent.

The agent can use more parameters to characterize the signal.
For example, the probabilities of~\(00\)s, \(01\)s, \(10\)s, and \(11\)s.
Using the same sequences, the agent would compute
\[
\begin{aligned}
&\probaof[\rlA]{00} = 0,\\
&\probaof[\rlA]{01} = \frac{1}{3},
\end{aligned}
\qquad\qquad
\begin{aligned}
&\probaof[\rlA]{10} = \frac{1}{3}, \\
&\probaof[\rlA]{11} = \frac{1}{3},
\end{aligned}
\]
for~\(S\param{\rlA}\), and
\[
\begin{aligned}
&\probaof[\rlB]{00} = \frac{1}{6},\\
&\probaof[\rlB]{01} = \frac{1}{6},
\end{aligned}
\qquad\qquad
\begin{aligned}
&\probaof[\rlB]{10} = \frac{1}{6}, \\
&\probaof[\rlB]{11} = \frac{1}{2},
\end{aligned}
\]
for~\(S\param{\rlB}\).
Using these parameters, which correspond to a deeper analysis, the agent is able to differentiate~\(S\param{\rlA}\) and~\(S\param{\rlB}\).
Sequences~\(S\param{\rlA}\) and~\(S\param{\rlB}\) are not depth-\(1\) consistent.
\end{example}

Consider~\(C\), an~\(\cS\)-valued process.
For~\(k\) in~\(\bN\), its depth-\(k\) characteristic~\(\chi\K\) is the long-run distribution of the strings of length~\(k + 1\).
For~\(d\) in~\(\cS^{k + 1}\)
\begin{equation*}
\chi\K\elmt{d} = \limfty{t} \probaof{\tuple{\seqqcom{C}{\tm}{t - k}{t - 1}{t}} = d}.
\end{equation*}
Two processes with the same depth-\(k\) characteristic are called depth-\(k\) consistent.


The signal observed by the agent is one such~\(\cS\)-valued process.
Consider another~\(\cS\)-valued process described by
\begin{subequations}
\label{eq:exogenous_mockup_system}
\begin{align}
\label{eq:exogenous_mockup_model}
z\nxt &= m\K \of{z, s\nxt}, \\
\label{eq:exogenous_mockup_predictor}
s\nxt &\drawn \mu \of{z},
\end{align}
\end{subequations}
where~\(z\) is a state in~\(\cS\pow{k}\) and~\(m\K\) is the length-\(k\)--memory function defined by
\[
m\K \of{\tuple{\seqqcom{s}{\tm}{t - k + 1}{t - 1}{t}}, s\Tp} = \tuple{\seqqcom{s}{\tm}{t - k + 2}{t}{t + 1}},
\]
and whose block diagram is depicted in~\cref{fig:block_diagram_mockup_exploded}.
Under some technical assumptions, described in~\cref{sec:empirical_evidence_optimality}, the observed signal and the Markov chain described by~\cref{eq:exogenous_mockup_system} are ergodic processes.
Furthermore, the Markov chain is depth-\(k\) consistent with the true signal when the following equality holds:
\[
\mu \of{z} \elmt{s\nxt} = \limfty{t}\probacond[\rbN, \sigma]{s\Tp = s\nxt}{\tuple{\seqqcom{s}{\tm}{t - k+ 1}{t - 1}{t}} = z}.
\]

\begin{figure}[htp]
\begin{subfigure}[b]{0.5\textwidth}
\centering
\inputtikz{block_diagram_mockup_exploded}
\caption{Full representation}
\end{subfigure}%
\begin{subfigure}[b]{0.5\textwidth}
\centering
\inputtikz{block_diagram_mockup}
\caption{Condensed representation}
\end{subfigure}
\ccaption[Block diagram for the depth-\(k\) mockup.]{
}
\label{fig:block_diagram_mockup_exploded}
\end{figure}

\begin{example}[Binary Signal {[continued]}]
The agent has observed a realization of the signal.
It has computed the parameters of interest.
The agent now creates a mockup with similar parameters.
It fixes, a priori, a deterministic Markov chain, as described by~\cref{eq:exogenous_mockup_model}.
Then, it identifies the values of~\(\mu\) making the mockup consistent with the observed sequence.

For depth-\(0\) consistency, it is sufficient to use a Markov chain with a single state, as depicted in~\cref{fig:consistency_one-state_markov_chain}.
Remember that sequences~\(S\param{\rlA}\) and~\(S\param{\rlB}\) are depth-\(0\) consistent.
Therefore, a single distribution~\(\mu \of{z\param{\emptyset}}\) makes the mockup depth-\(0\) consistent with~\(S\param{\rlB}\) and~\(S\param{\rlB}\).
This distribution is the following:
\[
\begin{aligned}
&\mu \of{z\param{\emptyset}} \elmt{0} = \probaof[\rlA]{0} = \probaof[\rlB]{0} = \frac{1}{3}, \\
&\mu \of{z\param{\emptyset}} \elmt{1} = \probaof[\rlA]{1} = \probaof[\rlB]{1} = \frac{2}{3}.
\end{aligned}
\]

\begin{figure}[htp]
\centering
\inputtikz{consistency_one-state_markov_chain}
\ccaption[Mockup enabling depth-\(0\) consistency with binary signals.]{
The mockup is composed of a fixed deterministic Markov chain~\(m^{0}\) with a single state and a distribution~\(\mu \of{z\param{\emptyset}}\).
By adjusting the distribution~\(\mu \of{z\param{\emptyset}}\), this mockup can be made depth-\(0\) consistent with any given ergodic binary signal.
}
\label{fig:consistency_one-state_markov_chain}
\end{figure}


For depth-\(1\) consistency, a Markov chain with two state, portrayed in~\cref{fig:consistency_two-state_markov_chain}, is required.
The distributions making the mockup depth-\(1\) consistent with the sequence~\(S\param{\rlA}\) are
\[
\begin{aligned}
&\mu \of{z\param{0}} \elmt{0} = \probacond[\rlA]{0}{z\param{0}} = 0, \\
&\mu \of{z\param{0}} \elmt{1} = \probacond[\rlA]{1}{z\param{0}} = 1,
\end{aligned}
\qquad\text{and}\qquad
\begin{aligned}
&\mu \of{z\param{1}} \elmt{0} = \probacond[\rlA]{0}{z\param{1}} = \frac{1}{2}, \\
&\mu \of{z\param{1}} \elmt{1} = \probacond[\rlA]{1}{z\param{1}} = \frac{1}{2}.
\end{aligned}
\]
Similarly,the distributions making the mockup depth-\(1\) consistent with the sequence~\(S\param{\rlB}\) are
\[
\begin{aligned}
&\mu \of{z\param{0}} \elmt{0} = \probacond[\rlB]{0}{z\param{0}} = \frac{1}{2}, \\
&\mu \of{z\param{0}} \elmt{1} = \probacond[\rlB]{1}{z\param{0}} = \frac{1}{2},
\end{aligned}
\qquad\text{and}\qquad
\begin{aligned}
&\mu \of{z\param{1}} \elmt{0} = \probacond[\rlB]{0}{z\param{1}} = \frac{1}{4}, \\
&\mu \of{z\param{1}} \elmt{1} = \probacond[\rlB]{1}{z\param{1}} = \frac{3}{4}.
\end{aligned}
\]

\begin{figure}[htp]
\centering
\inputtikz{consistency_two-state_markov_chain}
\ccaption[Mockup enabling depth-\(1\) consistency with binary signals.]{
The mockup is composed of a fixed deterministic Markov chain~\(m^{1}\) with two states and two distributions~\(\mu \of{z\param{0}}\) and~\(\mu \of{z\param{1}}\).
By adjusting the distributions~\(\mu \of{z\param{0}}\) and~\(\mu \of{z\param{1}}\), this mockup can be made depth-\(1\) consistent with any given ergodic binary signal.
}
\label{fig:consistency_two-state_markov_chain}
\end{figure}

Note that the mockup is split in two parts.
First, since the agent does not know anything about the signal, it assumes that it is generated by a parametric model.
Second, it computes observation distributions matching the empirical evidence provided by the signal.
The mockup built this way is consistent with the signal.
Nothing in the empirical evidence contradicts the assumption made by the agent.
\end{example}

Denote by~\(\rbM\K\) the dynamical system described by~\cref{eq:private_dynamic,eq:exogenous_mockup_system}.
This system is obtained from the original system~\(\rbN\) by substituting Nature with a depth-\(k\) consistent mockup.
It is depicted in~\cref{fig:block_diagram_single-agent_model}.
The system~\(\rbM\K\) induces an \ac{mdp} with state~\(\tuple{x, z}\), action~\(a\), strategy~\(\sigma \from \cX \times \cZ \to \cA\), and the objective function
\[
U\params{\rbM\K, \sigma}\of{x\tm{0}, z\tm{0}} = \expectof[\rbM\K, \sigma]{\sum\idxfromto{t}{0}{\infty} \delta^t u \of{x\Tt, a\Tt, s\Tp}}.
\]
A strategy~\(\sigma\) for the \ac{mdp} can be implemented in the real system by building~\(z\) with~\cref{eq:exogenous_mockup_model}.
This trick, allowing to use a strategy designed for the simpler system~\(\rbM\K\) in the more complicated system~\(\rbN\), is illustrated in \cref{fig:block_diagram_single-agent_problem_with_mockup}.
The agent knowledge in system~\(\rbM\K\) is depicted in~\cref{fig:agent_knowledge_single-agent_empirical_model}.
The simplification to the agent knowledge system~\(\rbN\), induced by the use of the length-\(k\)--memory function, is presented in~\cref{fig:agent_knowledge_single-agent_empirical_emphasized_depth_k}.

\begin{figure}[htp]
\begin{subfigure}{0.5\textwidth}
\centering
\inputtikz{block_diagram_single-agent_model}
\caption{Open loop}
\end{subfigure}%
\begin{subfigure}{0.5\textwidth}
\centering
\inputtikz{block_diagram_single-agent_model_with_strategy}
\caption{Closed loop with a strategy}
\end{subfigure}
\ccaption[Block diagram for the single-agent empirical-evidence setup with mockup~\(\rbM\K\).]{
For a given distribution~\(\mu\), the system forms a finite~\ac{mdp} and an optimal strategy can be computed.
}
\label{fig:block_diagram_single-agent_model}
\end{figure}

\begin{figure}[htp]
\begin{subfigure}{0.5\textwidth}
\centering
\inputtikz{block_diagram_single-agent_problem_with_mockup}
\caption{Open loop}
\end{subfigure}%
\begin{subfigure}{0.5\textwidth}
\centering
\inputtikz{block_diagram_single-agent_problem_with_mockup_and_strategy}
\caption{Closed loop with a strategy}
\end{subfigure}
\ccaption[Block diagram for the single-agent empirical-evidence setup~\(\rbN\) with length-\(k\)--memory function.]{
For a given strategy, the closed-loop system generates a probability distribution~\(\probacond{s\nxt}{z}\).
Using this distribution for~\(\mu\) along with~\(m\K\) yields a depth-\(k\)--consistent mockup.
}
\label{fig:block_diagram_single-agent_problem_with_mockup}
\end{figure}

\begin{figure}[htp]
\centering
\inputtikz{agent_knowledge_single-agent_empirical_model}
\ccaption[Agent knowledge in the single-agent empirical-evidence setup with mockup~\(\rbM\K\).]{
The agent knows all of the causality relations and all the parameters of the~\ac{mdp}, namely~\(f\) and~\(\mu\).
}
\label{fig:agent_knowledge_single-agent_empirical_model}
\end{figure}

\begin{sidewaysfigure}
\centering
\inputtikz{agent_knowledge_single-agent_empirical_emphasized_depthk}
\ccaption[Agent knowledge in the single-agent empirical-evidence setup~\(\rbN\), simplified by the use of the length-\(k\)--memory function.]{
The full private information is not retained anymore.
The pair~\(\tuple{x,z}\) is the information used by the agent to compute the next action.
}
\label{fig:agent_knowledge_single-agent_empirical_emphasized_depth_k}
\end{sidewaysfigure}

By using depth-\(k\) consistency, we went from an arbitrarily complicated system~\(\rbN\) to a finite~\ac{mdp}~\(\rbM\K\).
The next idea is to compute optimal strategies in the simpler system and measure their impact in the real system.

Consider the following iterative process, illustrated in \cref{fig:iterative_process}.
The agent implements an initial strategy~\(\sigma\tm{0}\).
It formulates a depth-\(k\) consistent model~\(\mu\tm{0}\) of Nature's dynamic.
Then, it computes an optimal strategy~\(\sigma\tm{1}\) for the \ac{mdp} induced by this model~\(\mu\tm{0}\).
Upon implementation of this new strategy, the model~\(\mu\tm{0}\) may lose the requisite statistical consistency.
Therefore, the agent formulates a revised depth-\(k\) consistent model~\(\mu\tm{1}\) and the process repeats.
A fixed point of this iterative process is one way to define a solution to this problem.
A strategy is a solution if it is optimal with respect to the model it induces.

\begin{figure}[pht]
\begin{subfigure}[b]{0.5\textwidth}
\centering
{
  \renewcommand{\sigmaparameter}{0}
  \renewcommand{\muparameter}{0}
  \inputtikz{block_diagram_single-agent_problem_with_mockup_and_strategy}
}
\caption{Use~\(\sigma\param{0}\) in~\(\rbN\) and measure~\(\mu\param{0}\) that makes the mockup depth-\(k\) consistent.}
\end{subfigure}%
\begin{subfigure}[b]{0.5\textwidth}
\centering
{
  \renewcommand{\sigmaparameter}{1}
  \renewcommand{\muparameter}{0}
  \inputtikz{block_diagram_single-agent_model_with_strategy}
}
\caption{Use~\(\mu\param{0}\) in~\(\rbM\K\) to setup \aac{mdp}.
Compute~\(\sigma\param{1}\) optimal for this \ac{mdp}.}
\end{subfigure}

\vspace{40pt}

\begin{subfigure}[b]{0.5\textwidth}
\centering
{
  \renewcommand{\sigmaparameter}{1}
  \renewcommand{\muparameter}{1}
  \inputtikz{block_diagram_single-agent_problem_with_mockup_and_strategy}
}
\caption{Use~\(\sigma\param{1}\) in~\(\rbN\) and measure~\(\mu\param{1}\) that makes the mockup depth-\(k\) consistent.}
\end{subfigure}%
\begin{subfigure}[b]{0.5\textwidth}
\centering
{
  \renewcommand{\sigmaparameter}{2}
  \renewcommand{\muparameter}{1}
  \inputtikz{block_diagram_single-agent_model_with_strategy}
}
\caption{Use~\(\mu\param{1}\) in~\(\rbM\K\) to setup \aac{mdp}.
Compute~\(\sigma\param{2}\) optimal for this \ac{mdp}.}
\end{subfigure}
\vspace{40pt}
\ccaption[Iterative process alternating between the real and the mockup system.]{
Using a strategy in the real system yields a measurement.
Using this measurement in the mockup system yields an optimal strategy.
}
\label{fig:iterative_process}
\end{figure}

Using that model to design a strategy is equivalent to the agent making an assumption about the system.
For example, when the agent uses a depth-\(k\) consistent model, it assumes the signal is generated exogenously, \ie, not impacted by~\(x\) or~\(a\).
This assumption might seem restrictive.
However, note that the repeated-modeling and optimization phases create a feedback loop.
Therefore, a model satisfying the consistency condition is exogenous but captures characteristics of Nature's dynamic.

The following section extends beyond the notion of depth-\(k\) consistency.

\section{Empirical-evidence Optimality}
\label{sec:empirical_evidence_optimality}

The agent assumes that a Markov chain, with state~\(z\) from a finite set~\(\cZ\), generates the signal~\(s\) and that it can construct~\(z\) from its observations as follows:
\begin{subequations}
\label{eq:mockup_system}
\begin{align}
\label{eq:mockup_model}
z\nxt &\drawn m \of{z, x\nxt, a, s\nxt}, \\
\label{eq:mockup_predictor}
s\nxt &\drawn \mu \of{z}.
\end{align}
\end{subequations}
The model~\(m\) represents the assumption the agent makes about the system.
The predictor~\(\mu\) is the set of parameters the agent adjusts to obtain a signal resembling its observations.
The pair~\(\mmu\) is called a mockup.
Denote by~\(\rbM\) the dynamical system described by \cref{eq:private_dynamic,eq:mockup_system}.
The block diagram for this system is depicted in \cref{fig:block_diagram_single-agent_full_mockup_with_strategy}.
The block diagram for the associated system~\(\rbN\) with the model~\(m\) is depicted in \cref{fig:block_diagram_single-agent_problem_with_full_mockup_and_strategy}.
The agent knowledge in systems~\(\rbM\) and~\(\rbN\) are presented in \cref{fig:agent_knowledge_single-agent_empirical_model_full} and \cref{fig:agent_knowledge_single-agent_empirical_emphasized_full_mockup} respectively.

\begin{figure}[htp]
\centering
\inputtikz{block_diagram_single-agent_full_mockup_with_strategy}
\ccaption[Block diagram for the single-agent empirical-evidence setup with mockup~\(\rbM\).]{
}
\label{fig:block_diagram_single-agent_full_mockup_with_strategy}
\end{figure}

\begin{figure}[htp]
\centering
\inputtikz{block_diagram_single-agent_problem_with_full_mockup_and_strategy}
\ccaption[Block diagram for the single-agent empirical-evidence setup~\(\rbN\) with model~\(m\).]{
}
\label{fig:block_diagram_single-agent_problem_with_full_mockup_and_strategy}
\end{figure}

\begin{figure}[htp]
\centering
\inputtikz{agent_knowledge_single-agent_empirical_model_full}
\ccaption[Agent knowledge in the single-agent empirical-evidence setup with mockup~\(\rbM\).]{
}
\label{fig:agent_knowledge_single-agent_empirical_model_full}
\end{figure}

\begin{sidewaysfigure}
\centering
\inputtikz{agent_knowledge_single-agent_empirical_emphasized_full_mockup}
\ccaption[Agent knowledge in the single-agent empirical-evidence setup~\(\rbN\), simplified by the use of the model~\(m\).]{
}
\label{fig:agent_knowledge_single-agent_empirical_emphasized_full_mockup}
\end{sidewaysfigure}

In this setup, depth-\(k\) consistency is replaced with the following definition.
\begin{definition}
\label{def:consistency}
Let~\(\sigma\) be a strategy and~\(\mmu\) be a mockup.
Predictor~\(\mu\) is~\(\sigmam\) consistent with~\(\rbN\) if
\[
\forall z \in \cZ \text{ and } s\nxt \in \cS\comma \mu \of{z} \elmt{s\nxt} = \limfty{t}\probacond[\rbN, \sigma]{s\Tp = s\nxt}{z\Tt = z}.
\]
\end{definition}

The notion of optimality used is the following.
\begin{definition}
Let~\(\sigma\) be a strategy, \(\mmu\) be a mockup, and~\(\epsilon\) be a positive number.
Strategy~\(\sigma\) is~\(\mum\) optimal if it is optimal for the \ac{mdp} induced by~\(\rbM\).
Strategy~\(\sigma\) is~\(\epsilonmum\)~optimal if it is~\(\epsilon\) optimal for the \ac{mdp} induced by~\(\rbM\).
\end{definition}

Having defined consistency and optimality the definition of an \ac{eeo} follows.
\begin{definition}
Let~\(\sigma\) be a strategy, \(\mmu\) be a mockup, and~\(\epsilon\) be a positive number.
The pair~\(\sigmamu\) is an~\(m\) \ac{eeo} if the following two conditions hold:
\begin{enumerate}
\item Strategy~\(\sigma\) is~\(\mum\) optimal.
\item Predictor~\(\mu\) is~\(\sigmam\) consistent with~\(\rbN\).
\end{enumerate}
The pair~\(\sigmamu\) is an~\(\epsilonm\) \ac{eeo} if the following two conditions hold:
\begin{enumerate}
\item Strategy~\(\sigma\) is~\(\epsilonmum\) optimal.
\item Predictor~\(\mu\) is~\(\sigmam\) consistent with~\(\rbN\).
\end{enumerate}
\end{definition}

A little care must be taken to make~\(\mu\) in~\cref{def:consistency} well defined.
Insuring the following assumption is verified guarantees it.
\begin{assumption}
\label{ass:ergodicity}
Let~\(\sigma\) be a strategy, and~\(T\strat{\sigma}\) be the Markov chain with state~\(X = \tuple{w, x, z}\) induced by~\(\rbN\) and~\(\sigma\), \(X\nxt \drawn T\strat{\sigma} X\).
The Markov chain~\(T\strat{\sigma}\) is irreducible and aperiodic.
In this case, some authors say that the Markov chain~\(T\strat{\sigma}\) is ergodic.
\end{assumption}

\Cref{ass:ergodicity} insures that~\(T\strat{\sigma}\) has a unique stationary distribution~\(\pi\strat{\sigma}\) such that
\[
\limfty{t}\probacond[\rbN, \sigma]{s\Tp = s\nxt}{z\Tt = z} = \probacond[\pi\strat{\sigma}]{s\nxt}{z}.
\]
Furthermore, \cref{ass:ergodicity} guarantees that~\(\pi\strat{\sigma}\) has full support, meaning that for all~\(w\) in~\(\cW\), \(x\) in~\(\cX\), and~\(z\) in~\(\cZ\), \(\pi\strat{\sigma}\elmt{w,x,z}\) is positive.
This guarantees that~\(\mu\) in~\cref{def:consistency} is well defined for all~\(z\) and~\(s\) as follows:
\begin{equation}
\label{eq:predictor_definition}
\begin{aligned}
\mu \of{z} \elmt{s\nxt}
&= \limfty{t}\probacond[\rbN, \sigma]{s\Tp = s\nxt}{z\Tt = z} \\
&= \probacond[\pi\strat{\sigma}]{s\nxt}{z} \\
&= \sum\idxin{w\nxt}{\cW} \probacond[\pi\strat{\sigma}]{s\nxt}{z, w\nxt} \mult \probacond[\pi\strat{\sigma}]{w\nxt}{z} \\
&= \sum\idxin{w\nxt}{\cW} \probacond[\pi\strat{\sigma}]{s\nxt}{w\nxt} \frac{\probaof[\pi\strat{\sigma}]{w\nxt, z\nxt}}{\probaof[\pi\strat{\sigma}]{z}} \\
&= \begin{multlined}[t][3.4in]
\sum\idxin{w\nxt}{\cW} \nu\of{w\nxt}\elmt{s\nxt} \mult \\\frac{\displaystyle \sum\idx{\substack{w \in \cW\\x \in \cX\\a \in \cA}} \pi\strat{\sigma}\elmt{w,x,z} \mult \sigma \of{x, z} \elmt{a} \mult n \of{w,x,a} \elmt{w\nxt}}{\displaystyle\sum\idx{\substack{w \in \cW\\ x\in\cX}} \pi\strat{\sigma}\elmt{w,x,z}}
\end{multlined}
\end{aligned}
\end{equation}

One way to insure that~\cref{ass:ergodicity} is verified is to have a small noise affect all the transitions.
Formally, this means that for all~\(w \in \cW\), \(x \in \cX\), \(a \in \cA\), and~\(s\nxt \in \cS\), \(f \of{x, a, s\nxt}\), \(n \of{w, x, a}\), \(\nu \of{w}\), and~\(\sigma \of{x, z}\) have full support.
From now on, \cref{ass:ergodicity} is always verified.

\section{Weak Consistency and Eventual Consistency}
The notion of consistency exposed in \cref{def:consistency} is fairly strong.
It requires that the quantity~\(\probacond[\rbN, \sigma]{s\Tp = s\nxt}{z\Tt = z}\) converges for all~\(z \in \cZ\) and~\(s\nxt \in \cS\).
As a result, the associated \cref{ass:ergodicity} is constraining.
This section, highlights two slightly less stringent notions of consistency with their associated assumptions.
The first one, weak consistency, uses a weaker notion of convergence.
The second one, eventual consistency, only requires convergence on a subset of states~\(z \in \cZ\).

First, let us define the notion of weak consistency.
It relies on the fact that convergence of the average of a sequence is weaker than convergence of the sequence.

\begin{definition}
\label{def:weak_consistency}
Let~\(\sigma\) be a strategy and~\(\mmu\) be a mockup.
Predictor~\(\mu\) is weakly \(\sigmam\) consistent with~\(\rbN\) if
\[
\forall z \in \cZ \text{ and } s\nxt \in \cS\comma \mu \of{z} \elmt{s\nxt} = \limfty{T} \frac{1}{T} \sum\idxfromto{t}{1}{T} \probacond[\rbN, \sigma]{s\Tp = s\nxt}{z\Tt = z}.
\]
\end{definition}

The associated assumption is the following.

\begin{assumption}
\label{ass:weak_ergodicity}
Let~\(\sigma\) be a strategy, and~\(T\strat{\sigma}\) be the Markov chain with state~\(X = \tuple{w, x, z}\) induced by~\(\rbN\) and~\(\sigma\), \(X\nxt \drawn T\strat{\sigma} X\).
The Markov chain~\(T\strat{\sigma}\) is irreducible.
\end{assumption}
\Cref{ass:weak_ergodicity} insures that~\(T\strat{\sigma}\) has a unique stationary distribution~\(\pi\strat{\sigma}\) such that
\[
\limfty{T} \frac{1}{T} \sum\idxfromto{t}{1}{T} \probacond[\rbN, \sigma]{s\Tp = s\nxt}{z\Tt = z} = \probacond[\pi\strat{\sigma}]{s\nxt}{z}.
\]
Furthermore, \cref{ass:weak_ergodicity} guarantees that~\(\pi\strat{\sigma}\) has full support.
Using the same reasoning as in \cref{eq:predictor_definition} guarantees that~\(\mu\) in~\cref{def:weak_consistency} is well defined.

Second, let us define the notion of eventual consistency.
It relies on the fact that the value of~\(\mu \of{z}\) for a state~\(z\) with zero probability in the limit is irrelevant.

\begin{definition}
\label{def:eventual_consistency}
Let~\(\sigma\) be a strategy and~\(\mmu\) be a mockup.
Predictor~\(\mu\) is eventually \(\sigmam\) consistent with~\(\rbN\) if for all states~\(z \in \cZ\) such that \(\limfty{t} \probaof[\rbN, \sigma]{z\Tt = z} > 0\)
\[
\forall s\nxt \in \cS\comma \mu \of{z} \elmt{s\nxt} = \limfty{t} \probacond[\rbN, \sigma]{s\Tp = s\nxt}{z\Tt = z}.
\]

For~\(z \in \cZ\) such that~\(\limfty{t} \probaof[\rbN, \sigma]{z\Tt = z} = 0\), there is no requirement on \(\mu \of {z}\).
Its value is totally arbitrary.
\end{definition}

If a state~\(z\) is such that~\(\limfty{t} \probaof[\rbN, \sigma]{z\Tt = z} = 0\), it will not be seen in the long run.
Therefore, there is no need to impose some constraints on~\(\mu\) for this state.
The following assumption is associated with this definition.

\begin{assumption}
\label{ass:eventual_ergodicity}
Let~\(\sigma\) be a strategy, and~\(T\strat{\sigma}\) be the Markov chain with state~\(X = \tuple{w, x, z}\) induced by~\(\rbN\) and~\(\sigma\), \(X\nxt \drawn T\strat{\sigma} X\).
The Markov chain~\(T\strat{\sigma}\) is unichain and aperiodic.
\end{assumption}

A Markov chain is unichain if it contains a single communication class.
\Cref{ass:eventual_ergodicity} insures that~\(T\strat{\sigma}\) has a unique stationary distribution~\(\pi\strat{\sigma}\) with full support on its communication class, and no support outside of it.
Therefore, for~\(z \in \cZ\) such that~\(\limfty{t} \probaof[\rbN, \sigma]{z\Tt = z} > 0\), it is the case that~\(\probaof[\pi\strat{\sigma}]{z} > 0\).
Furthermore, for that same~\(z\),
\[
\limfty{t} \probacond[\rbN, \sigma]{s\Tp = s\nxt}{z\Tt = z} = \probacond[\pi\strat{\sigma}]{s\nxt}{z}.
\]
Once again, using the same reasoning as in \cref{eq:predictor_definition} guarantees that~\(\mu\) in~\cref{def:eventual_consistency} is well defined.
The division by~\(\probaof[\pi\strat{\sigma}]{z}\) only occurs when this quantity is strictly positive.
When it is zero, \(\mu\) is defined arbitrarily.

Combining~\cref{def:weak_consistency,def:eventual_consistency} yields a third notion, eventual weak consistency.
The associated assumption requires the Markov chain to be unichain.

The following list summarizes the different notions of consistency and their associated assumptions on the Markov chain:
\begin{description}
\item[Consistent] Irreducible and aperiodic.
\item[Weakly consistent] Irreducible.
\item[Eventually consistent] Unichain and aperiodic.
\item[Eventually weakly consistent] Unichain.
\end{description}

For conciseness reasons, the rest of this presentation only mentions the first notion of consistency defined in \cref{def:consistency}.
However, the results are applicable to all the notions of consistency defined in the present section.
Extra care needed to accommodate a specific notion of consistency will be addressed on a case by case basis.

\section{Predictors and Strategies}
\label{sec:predictors_and_strategies}

Given a strategy~\(\sigma\), there is a unique predictor~\(\mu\) which is~\(\sigmam\) consistent with~\(\rbN\).
This predictor can be measured in the system~\(\rbN\), as depicted in \cref{fig:block_diagram_single-agent_problem_with_full_mockup_and_strategy}.
Note that~\cref{eq:predictor_definition} guarantees that~\(\mu\) is a continuous function of~\(\sigma\) and~\(\pi\strat{\sigma}\).
The mapping associating a predictor to each strategy is a function.
This function is denoted by~\(\FM\), where~\(\rM\) stands for modeling.

Given a predictor~\(\mu\), there might be multiple strategies~\(\sigma\) that are~\(\mum\) optimal.
Such strategies are the optimal strategies for the \ac{mdp} induced by system~\(\rbM\), depicted in \cref{fig:block_diagram_single-agent_full_mockup_with_strategy}.
Therefore, the mapping associating a predictor to the corresponding optimal strategies is a correspondence.
This correspondence is denoted~\(\FO\), where~\(\rO\) stands for optimization.
As in~\cref{chap:static_game_theory}, we define a function approximating this correspondence.
This will allow us later to once again use Brouwer's fixed-point theorem to gain intuition before using Kakutani's.
Consider the \ac{mdp} induced by~\(\rbM\).
Let~\(\Uopt : \cX \times \cZ \to \bR\) be the value function for that \ac{mdp}.
Define~\(Q : \cX \times \cZ \times \cA \to \bR\) by
\[
Q \of{x, z, a} = \expectof[s\nxt \drawn \nu \of{z}]{u \of{x, a, s\nxt} + \delta \expectcond[\rbM]{\Uopt \of{x\nxt, z\nxt}}{x, z, a, s\nxt}},
\]
and~\(\sigma\) by
\[
\sigma \of{x, z} \elmt{a} = \gibbsQ.
\]
The astute reader recognizes the Gibbs distribution, described in details in~\cref{note:gibbs_distribution}.
Recall that, as~\(\tau\) goes to~\(0\), \(\sigma\) converges to a~\(\mum\)-optimal strategy.
When~\(\tau\) is small enough, \(\sigma\) is~\(\epsilonmum\) optimal.
To guarantee uniqueness, define~\(\tau\) to be the largest value such that~\(\sigma\) is~\(\epsilonmum\) optimal.
Note that~\(\sigma\) defined that way is a continuous function of the value function~\(\Uopt\).
This function approximating the optimization correspondence is denoted~\(\FOe\).

The composition of an optimization mapping and a modeling mapping gives a mapping from the space of strategies to itself.
Two such mappings can be defined, the correspondence \(\FF = \FO \compo \FM\) and the function \(\FFe = \FOe \compo \FM\).

The following subsection extends the notion of \acp{eeo} to the multiagent case and defines \acp{eee}.

\section{Multiagent Setup}

Consider a collection of agents~\(\cI\).
Each agent~\(i\) has a state~\(x\Ii\), an action~\(a\Ii\), and a signal~\(s\Ii\).
Let~\(x\) be the tuple~\(\tuple{\sseqcom{x}{\ag}{1}{2}{\abs{\cI}}}\).
Define~\(a\) and~\(s\) similarly.
Agent~\(i\) is controlling the system described by
\begin{equation}
\label{eq:private_dynamic_i}
x\Ii\nxt \drawn  f\Ii \of{x\Ii, a\Ii, s\Ii\nxt}.
\end{equation}
Agents~\(-i\) are controlling systems described as a whole by
\begin{equation}
\label{eq:private_dynamic_-i}
x\mI\nxt \drawn  f\mI \of{x\mI, a\mI, s\mI\nxt}.
\end{equation}
All these systems are coupled through Nature which determines the signals~\(s\) according to
\begin{subequations}
\label{eq:nature_system_i}
\begin{align}
\label{eq:nature_dynamic_i}
w\nxt &\drawn n \of{w, x, a}, \\
\label{eq:nature_signal_i}
s\nxt &\drawn \nu \of{w\nxt}.
\end{align}
\end{subequations}

The rest of this section extends the notions of consistency and optimality to this setting.
The block diagrams and agent-knowledge figures for a two-agent setup, available in~\cref{fig:block_diagram_two-agent_problem,fig:block_diagram_two-agent_problem_with_full_mockup_and_strategy,fig:agent_knowledge_two-agent,fig:agent_knowledge_two-agent_model}, help in following the discussion.
In particular, they highlight the fact that there are very few differences in the treatment of the single-agent problem and the multiagent one.
By design, the notions of consistency and optimality used remove the differences between the two settings.

\begin{figure}[htp]
\begin{subfigure}{\textwidth}
\centering
{
  \renewcommand{\agparameter}{1}
  \inputtikz{agent_knowledge_single-agent_empirical_model_full}
}
\caption{Agent~\(1\)'s knowledge in~\(\rbM\one\)}

\end{subfigure}

\vspace{40pt}

\begin{subfigure}{\textwidth}
\centering
{
  \renewcommand{\agparameter}{2}
  \inputtikz{agent_knowledge_single-agent_empirical_model_full}
}
\caption{Agent~\(2\)'s knowledge in~\(\rbM\two\)}
\end{subfigure}%

\vspace{40pt}

\ccaption[Agent knowledge in the two-agent empirical-evidence setup with mockup~\(\rbM\one\) and \(\rbM\two\).]{
Once again, the diagrams for the two agents are identical and not different from the single agent setting.
}
\label{fig:agent_knowledge_two-agent_model}
\end{figure}

\begin{sidewaysfigure}
\centering
\inputtikz{agent_knowledge_two-agent}
\ccaption[Agent knowledge for the two-agent empirical-evidence setup~\(\rbN\) with models~\(m\one\) and~\(m\two\).]{
Agent~\(1\)'s knowledge of the causality relations is highlighted with bold arrows.
Notice that, from agent~\(1\)'s point of view, nothing has changed from the single-agent setup.
Its signal is underspecified but it does not realize that there is another agent along with Nature.
}
\label{fig:agent_knowledge_two-agent}
\end{sidewaysfigure}

\begin{figure}[htp]
\centering
\inputtikz{block_diagram_two-agent_problem}
\ccaption[Block diagram for the two-agent empirical-evidence setup~\(\rbN\) with models~\(m\one\) and~\(m\two\).]{
}
\label{fig:block_diagram_two-agent_problem}
\end{figure}

\begin{figure}[htp]
\begin{subfigure}[b]{0.5\textwidth}
\centering
{
  \renewcommand{\agparameter}{1}
  \inputtikz{block_diagram_single-agent_full_mockup_with_strategy}
}
\caption{Agent~\(1\)'s block diagram}
\end{subfigure}%
\begin{subfigure}[b]{0.5\textwidth}
\centering
{
  \renewcommand{\agparameter}{2}
  \inputtikz{block_diagram_single-agent_full_mockup_with_strategy}
}
\caption{Agent~\(2\)'s block diagram}
\end{subfigure}%
\ccaption[Block diagrams in the two-agent empirical-evidence setup with mockup~\(\rbM\one\) and \(\rbM\two\).]{
Notice that the two block diagrams are identical.
Furthermore, from the agents' perspective, there is no difference between the single-agent and the multiagent setting.
}
\label{fig:block_diagram_two-agent_problem_with_full_mockup_and_strategy}
\end{figure}

Denote by~\(\rbN\Ii\) the system from agent~\(i\)'s perspective.
In the single-agent setup, \(\rbN\) was composed of a known part \cref{eq:private_dynamic} and an unknown part \cref{eq:nature_system}.
Similarly, \(\rbN\Ii\) has a known part~\cref{eq:private_dynamic_i} and an unknown part~ \cref{eq:private_dynamic_-i,eq:nature_system_i}.

The other definitions from previous sections can readily be extended to the multiagent case.
Agent~\(i\) has a utility function~\(u\Ii\), a discount factor~\(\delta\Ii\), a strategy~\(\sigma\Ii : \cP\Ii \to \distribover{\cA\Ii}\), and a mockup of Nature and its opponents described by a state~\(z\Ii\), a model~\(m\Ii\), and a predictor~\(\mu\Ii\).
This forms the system~\(\rbM\Ii\).

From agent~\(i\)'s perspective, everything is identical to the single-agent setup.
The notions of~\(\mum\) optimality, \(\epsilonmum\) optimality, and~\(\sigmam\) consistency can be replaced by~\(\muImI\) optimality, \(\epsilonImuImI\) optimality, and~\(\sigmamI\) consistency respectively.
Therefore, the definition of \ac{eeo} readily extends to the multiagent setting.

\begin{definition}
\label{def:eee}
Let~\(\sigma\), \(\mmu\), and~\(\epsilon\) such that for all~\(i\) in~\(\cI\), \(\sigma\Ii\) is a strategy, \(\mImuI\) is a mockup, and~\(\epsilon\Ii\) is a positive number.
The pair~\(\sigmamu\) is an~\(m\) \ac{eee} if the following two conditions hold for all~\(i\) in~\(\cI\):
\begin{enumerate}
\item Strategy~\(\sigma\Ii\) is~\(\muImI\) optimal.
\item Predictor~\(\mu\Ii\) is~\(\sigmamI\) consistent with~\(\rbN\).
\end{enumerate}
The pair~\(\sigmamu\) is an~\(\epsilonm\) \ac{eee} if the following two conditions hold for all~\(i\) in~\(\cI\):
\begin{enumerate}
\item Strategy~\(\sigma\Ii\) is~\(\epsilonImuImI\) optimal.
\item Predictor~\(\mu\) is~\(\sigmamI\) consistent with~\(\rbN\).
\end{enumerate}
\end{definition}

For a given~\(m\) and~\(\epsilon\) such that for all~\(i\) in~\(\cI\), \(\epsilon\Ii\) is a positive number, denote by~\(\FO\) the optimization correspondence from predictors to strategies, by~\(\FOe\) its approximating function, and by~\(\FM\) the modeling mapping from strategies to  predictors.

These mappings are defined by direct extension of their single agent counterparts.
Define~\(\FF\), a correspondence from the space of strategies to itself, by~\(\FF = \FO \compo \FM\).
Similarly define~\(\FFe\), a function from the space of strategies to itself, by~\(\FFe = \FOe \compo \FM\).

It is sometimes easier to work with a function from the space of predictors to itself.
In these cases, we use~\(\GGe = \FM \compo \FOe\).

Now that the setup has been established, it is time to prove some results.
The first result tackles the existence of \acp{eee}.

\section{Existence of Empirical-evidence Equilibria}

The proof of existence of empirical-evidence equilibria follows along the line of the proof of existence of Nash equilibria presented in \cref{sec:nash_existence_theorem}.
It starts by proving the existence of approximate equilibria through Brouwer's fixed-point theorem, before proving the existence of exact equilibria using Kakutani's fixed-point theorem.
The proof of the existence of Nash equilibria used the best-response correspondence.
The best response correspondence mapped the set of independent mixed actions to itself.
In the empirical-evidence setting, these actions are replaced by strategies, and~\(\FF\) plays a similar role to the best-response correspondence.

\subsection{Existence of Approximate Equilibria}

\begin{theorem}
\label{res:mme_existence}
Let~\(m = \agstuple{m}\) be models and~\(\epsilon =\agstuple{\epsilon} \) be positive numbers.

There exists an~\(\epsilonm\) \ac{eee}.
\end{theorem}

\begin{proof}
First, show that~\(\FFe\) has a fixed point.
The set of strategies is representable by a product of simplices.
Therefore, \(\FFe\) is a mapping from a convex and compact set to itself.
By \cref{res:modeling_continuity,res:optimization_continuity}, \(\FOe\) and~\(\FM\) are continuous.
As the composition of two continuous functions, \(\FFe\) is continuous.
By application of Brouwer's fixed-point theorem, \(\FFe\) has a fixed point.

The upcoming~\cref{res:fixed_point_EEE} therefore implies that an~\(\epsilonm\) \ac{eee} exists.
\end{proof}

\begin{proposition}
\label{res:fixed_point_EEE}
Let~\(m = \agstuple{m}\) be models and~\(\epsilon =\agstuple{\epsilon} \) be positive numbers.
Let~\(\sigma\opt\) be a fixed point of~\(\FFe\).
Define~\(\mu\opt\) by~\(\mu\opt = \FM \of{\sigma\opt}\).

The pair~\(\tuple{\mu\opt, \sigma\opt}\) is an~\(\epsilonm\) \ac{eee}.
\end{proposition}

\begin{proof}
Fix~\(i \in \cI\).
By definition  predictor~\(\mu\opt\) is~\(\sigmaoptmI\) consistent with~\(\rbN\Ii\).
Note that
\[\FOe \of{\mu\opt} = \FOe \compo \FM \of{\sigma\opt} = \FFe \of{\sigma\opt} = \sigma\opt.
\]
This implies that strategy~\(\sigma\opt\) is~\(\epsilonImuoptImI\) optimal.
Therefore, \(\tuple{\mu\opt, \sigma\opt}\) is an~\(\epsilonm\)~\ac{eee}.
\end{proof}

\begin{proposition}
\label{res:optimization_continuity}
Let~\(m = \agstuple{m}\) be models and~\(\epsilon =\agstuple{\epsilon} \) be positive numbers.

The optimization mapping~\(\FOe\) is continuous.
\end{proposition}

\begin{proof}
Agent~\(i\)'s predictor only affects agent~\(i\)'s strategy.
Therefore, proving that~\(\FOe\) is continuous, only requires showing that~\(\FOe\Ii: \mu\Ii \mapsto \sigma\Ii\) is continuous for all~\(i \in \cI\).
Decomposing this function as follows:
\[
\FOe\Ii: \mu\Ii \xmapsto{\subfunc{a}} U\Ii\opt \xmapsto{\subfunc{b}} \sigma\Ii,
\]
it is sufficient to prove that~\(\subfunc{a}\) and~\(\subfunc{b}\) are continuous.

The upcoming~\cref{res:markov_continuity} shows that the value function of a finite \ac{mdp} is a continuous function of the parameters of the problem.
Since~\(\mu\Ii\) is one of the parameters of the \ac{mdp} whose value function is~\(U\Ii\opt\), \(\subfunc{a}\) is continuous.
It was noted in~\cref{sec:predictors_and_strategies} that~\(\subfunc{b}\) is continuous.
\end{proof}

\begin{proposition}
\label{res:modeling_continuity}
Let~\(m = \agstuple{m}\) be models.

The modeling mapping~\(\FM\) is continuous.
\end{proposition}

\begin{proof}
Agent~\(i\)'s strategy impacts all the agents' predictors.
Proving the continuity of~\(\FM\), requires showing that~\(\FM\ag{i, j}: \sigma\Ii \mapsto \mu\ag{j}\) is continuous for all~\(i, j \in \cI\).
Decomposing this function as follows:
\[\FM\ag{i, j}: \sigma\Ii \xmapsto{\subfunc{c}} T\strat{\sigma} \xmapsto{\subfunc{d}} \pi\strat{\sigma} \xmapsto{\subfunc{e}} \mu\ag{j},
\]
it is sufficient to prove that~\(\subfunc{c}\), \(\subfunc{d}\), and~\(\subfunc{e}\) are continuous.

The elements~\(\tuple{\sigma\Ii \of{x\Ii, z\Ii}}\idx{x\Ii \in \cX\Ii, z\Ii, \in \cZ\Ii}\)are entries in the matrix~\(T\strat{\sigma}\).
Therefore~\(\subfunc{c}\) is linear, hence continuous.
\cite[Theorem~4.1]{meyer:1980} shows that the stationary distribution of a finite irreducible Markov chain is a continuous function of the elements of its transition matrix, which proves that~\(\subfunc{d}\) is continuous.
It was noted in~\cref{sec:predictors_and_strategies} that~\(\subfunc{e}\) is continuous.
\end{proof}

The result in~\cite[Theorem~4.1]{meyer:1980} targets finite irreducible Markov chains.
Therefore, the proof of \cref{res:modeling_continuity} immediately holds when using weak consistency.
The following observation allows to accommodate eventual consistency as well.
Consider a finite unichain Markov chain with communication class~\(\cC\).
Label the states such that the states in~\(\cC\) come before the other ones.
Let~\(T\) be the transition matrix for this Markov chain.
It has the following block structure:
\[
\begin{pmatrix}
  T' & 0 \\
  A  & B
\end{pmatrix}
\]
where~\(T'\) is the transition matrix for an irreducible Markov chain on~\(\cC\).
Denote by~\(\pi\) and~\(\pi'\) the stationary distributions of~\(T\) and~\(T'\).
These stationary distributions coincide on~\(\cC\), which in block notation is denoted by~\(\pi = \tuple{\pi' 0}\).
First, the function~\(T \mapsto T'\) is a projection and therefore continuous.
Then, the function~\(T' \mapsto \pi'\) is continuous according to~\cite[Theorem~4.1]{meyer:1980}.
Finally, the function~\(\pi' \mapsto \pi = \tuple{\pi' 0}\) is trivially continuous.
This guarantees that \cref{res:modeling_continuity} holds when using eventual consistency or eventual weak consistency.

\begin{lemma}[Continuity of the Value Function in the Parameters of \aac{mdp}]
\label{res:markov_continuity}
Consider a finite \ac{mdp} described by a dynamic~\(x\nxt \drawn f \of{x, a}\), a utility function~\(u \of{x, a}\), and a discount factor~\(\delta\).
Denote by~\(\theta\) the finite vector of parameters of the problem.
It corresponds to all the entries in~\(f\) and~\(u\).
Let~\(\Btheta\) be the Bellman operator associated with the problem.
By definition, the value function of the problem~\(\Utheta\) is the fixed point of~\(\Btheta\), \(\Utheta = \Btheta \Utheta\).

The function~\(\theta \mapsto \Utheta\) is continuous.
\end{lemma}

\begin{proof}
Let~\(\theta\) and~\(\theta'\) be two vectors of parameters corresponding to two \acp{mdp}.
Let~\(\Utheta\) and~\(\Uthetap\) be the value functions associated with these \acp{mdp}.
We will show that~\(\Utheta\) converges to~\(\Uthetap\) as~\(\theta\) converges to~\(\theta'\) by showing that~\(\norm{\Utheta - \Uthetap}\) converges to~\(0\).

The value function~\(\Utheta\) is a fixed point of~\(\Btheta\). The Bellman operator~\(\Btheta\) is a contraction mapping with Lipschitz constant~\(\delta\). As a result,
\begin{equation}
\label{eq:bellman_operator_contraction}
\begin{aligned}
\norm{\Utheta - \Uthetap} &= \norm{\Btheta \Utheta - \Uthetap} \\
&\le \norm{\Btheta \Utheta - \Btheta \Uthetap} + \norm{\Btheta\Uthetap - \Uthetap} \\
&\le \delta \norm{\Utheta - \Uthetap} + \norm{\Btheta\Uthetap - \Uthetap} \\
&\le \frac{1}{\onem{\delta}} \norm{\Btheta\Uthetap - \Uthetap}.
\end{aligned}
\end{equation}

We will now prove that the function~\(\theta \mapsto \Btheta \Uthetap\) is continuous because each of its finitely many components is continuous.
By definition, \(\parentheses{\Btheta \Uthetap} \of{x} = \max\idxin{a}{\cA} v \of{x, a, \theta}\), where~\(v \of{x, a, \theta} = u \of{x, a} + \delta f \of{x, a}\tr \Uthetap\).
For fixed~\(x\) and~\(a\), \(\theta \mapsto v \of{x, a, \theta}\) is linear and therefore continuous.
For a fixed~\(x\), \(\theta \mapsto \Btheta \Uthetap \of{x}\) is the maximum of a finite number of continuous functions and as such is continuous.
Therefore, the function~\(\theta \mapsto \Btheta \Uthetap\) is continuous.

As a result, as~\(\theta\) converges to~\(\theta'\), \(\Btheta \Uthetap\) converges to~\(\Bthetap \Uthetap\).
Since~\(\Uthetap\) is a fixed point of~\(\Bthetap\), \(\Btheta \Uthetap\) converges to~\(\Uthetap\).
We have proven that the limit of~\({\Btheta\Uthetap - \Uthetap}\) as~\(\theta\) goes to~\(\theta'\) is zero.
Finally, \cref{eq:bellman_operator_contraction} implies that~\(\norm{\Utheta - \Uthetap}\) goes to zero as~\(\theta\) goes to~\(\theta'\) which concludes the proof.
\end{proof}

\subsection{Existence of Exact Equilibria}

\begin{theorem}
\label{res:eee_existence}
Let~\(m = \agstuple{m}\) be models.

There exists an exact \(m\)~\ac{eee}.
\end{theorem}

\begin{proof}
This proof follows closely the proof of the existence of an exact Nash equilibrium.

An exact optimization counterpart to \cref{res:fixed_point_EEE} is easily established.
It guarantees that strategies~\(\sigma \in \Sigma\) forming a fixed point of~\(\FF\), corresponds to an \(m\)~\ac{eee}.
Therefore, proving the existence of such a fixed point is a sufficient condition to proving the theorem.

To apply Kakutani's fixed point theorem we need to prove the four following facts:
\begin{itemize}
\item The set~\(\Sigma\) is non-empty, compact and a convex subset of an Euclidean space.
\item For~\(\sigma \in \Sigma\), \(\FF \of {\sigma}\) is non-empty.
\item For~\(\sigma \in \Sigma\), \(\FF \of {\sigma}\) is convex.
\item The~\(\FF\) correspondence has a closed graph.
\end{itemize}

The first fact was already proven.
The following two are immediate application of dynamic-programming results for finite \acp{mdp} with discounted cost.
The definition of the set \(\FF \of {\sigma}\) through the Bellman equation guarantees its non-emptiness and convexity.

Let us prove that the~\(\FF\) correspondence has a closed graph.
The function~\(\FM\) is continuous.
The composition of a continuous function with a correspondence with a closed graph is also a correspondence with a closed graph.
Therefore, it is sufficient to show that the~\(\FO\) correspondence has a closed graph.
Let~\(\sigma = \seqin{\sigma}{\tm}{t}{\bN}\) and~\(\mu = \seqin{\mu}{\tm}{t}{\bN}\) be sequences of strategies and predictors such that for all~\(t\) in~\(\bN\), \(\sigma\Tt \in \FO \of {\mu\Tt}\).
Suppose that~\(\sigma\) converges to~\(\sigma\opt\) and~\(\mu\) converges to~\(\mu\opt\).
Let~\(i\) be an agent, \(x\Ii \in \cX\Ii\), and~\(z\Ii \in \cZ\Ii\).
For~\(t \in \bN\), the fact that~\(\sigma\Tt \in \FO \of {\mu\Tt}\) implies that~\(\sigma\Ii\Tt \in \FO\Ii \of {\mu\Ii\Tt}\).
This translates to
\[
\sigma\Ii\Tt \of{x\Ii, z\Ii} \in \argmax\idxin{a\Ii}{\cA\Ii} \expectsym \underdist{s\Ii\nxt \drawn \mu\Ii\Tt \of{z\Ii}} \begin{multlined}[t][2.5in] \Biggl\lbrack u\Ii\of{x\Ii,a\Ii,s\Ii\nxt} + \\\expectcond{U\param{\mu\Ii\Tt} \of{x\Ii\nxt, z\Ii\nxt}}{x\Ii,z\Ii,a\Ii, s\Ii\nxt} \Biggr\rbrack\end{multlined},
\]
where~\(U\param{\mu\Ii\Tt}\) is the value function of the~\ac{mdp} induced by~\(\mu\Ii\Tt\).
The arguments used in the proof of the existence of an approximate equilibrium show that the right-hand side is a continuous function of~\(\mu\Ii\Tt\).
Therefore, in the limit,
\[
\sigma\Ii\opt \of{x\Ii, z\Ii} \in \argmax\idxin{a\Ii}{\cA\Ii} \expectsym \underdist{s\Ii\nxt \drawn \mu\Ii\opt \of{z\Ii}} \begin{multlined}[t][2.5in] \Biggl\lbrack u\Ii\of{x\Ii,a\Ii,s\Ii\nxt} + \\ \expectcond{U\param{\mu\Ii\opt} \of{x\Ii\nxt, z\Ii\nxt}}{x\Ii,z\Ii,a\Ii,s\Ii\nxt} \Biggr\rbrack\end{multlined},
\]
Which proves that~\(\sigma\Ii\opt \in \FO\Ii \of{\mu\Ii\opt}\).
This fact is true for any agent and therefore~\(\FO\) has a closed graph.

Everything is now in place to apply Kakutani's fixed-point theorem and to conclude that an exact \(m\)~\ac{eee} always exists.
\end{proof}

This is the second time Brouwer's theorem is used before using Kakutani's to prove the existence of an exact equilibrium.
This two-step process is centered around the continuity required by Brouwer's theorem.
This continuity guarantees that, when a sequence of strategies is optimal with respect to a sequence of parameters, if both sequences converge, then the optimality is still true in the limit.
There is not much to gain by using directly Kakutani's theorem as the core of the proof is shared with Brouwer's.

The following section characterizes empirical-evidence equilibria in the setting of perfect-monitoring repeated games.

\section[xEEEs in Perfect-monitoring Repeated Games]{Exogenous Empirical-evidence Equilibria in Perfect-monitoring Repeated Games}

Having defined a new equilibrium concept, we wanted to compare it with existing ones.
We focused on repeated games as these are the most studied stochastic games.
Along the way, we found a partial characterization of~\acp{eee} in perfect-monitoring repeated games.

Consider a two-player perfect-monitoring repeated game with utilities \(u\one \from \cA \to \bR\), \(u\two \from \cA \to \bR\) and discount factors~\(\delta\one\),~\(\delta\two\).
In this game, the agents are using exogenous models~\(m\one\) and~\(m\two\).
Since there is no state in a repeated game, it means that~\(m\Ii\) only depends on~\(z\Ii\) and~\(s\Ii\nxt\) but not~\(a\Ii\).
The associated agent knowledge diagrams are depicted in~\cref{fig:initial}.

\begin{figure}[pht]
\centering
\begin{subfigure}{\textwidth}
\centering
\inputtikz{agent_knowledge_initial}
\caption{Classical repeated game setting}
\end{subfigure}%
\vspace{40pt}
\begin{subfigure}{\textwidth}
\centering
\inputtikz{agent_knowledge_real}
\caption{System~\(\rbN\) in the exogenous empirical-evidence setting}
\end{subfigure}%
\vspace{40pt}
\begin{subfigure}{\textwidth}
\centering
\inputtikz{agent_knowledge_model}
\caption{System~\(\rbM\one\) in the exogenous empirical-evidence setting}
\end{subfigure}%
\vspace{40pt}
\ccaption[Agent knowledge in a two-player perfect-monitoring repeated game.]{
The highlighted signal highlights how the consistency condition ties~\(\rbN\) and~\(\rbM\one\) together.
}
\label{fig:initial}
\end{figure}

The following proposition shows that, in the present setting, optimality is achievable with a strategy using only the last value of the model state instead of the whole history.

\begin{proposition}
\label{res:myopic_optimality}
Let~\(i\) be an agent,~\(\mu\Ii \from \cZ\Ii \to \distribover{\cA\mI}\) be a predictor, and \(\sigma\Ii \from \cZ\Ii \to \distribover{\cA\Ii}\) be a strategy such that the following condition holds:
\[
\forall z\Ii \in \cZ\Ii \comma
\forall a\Ii' \in \cA\Ii \comma
\expectof{u\Ii \of{\sigma\Ii \of{z\Ii}, \mu\Ii \of{z\Ii}}}
\ge
\expectof{u\Ii \of{a\Ii', \mu\Ii \of{z\Ii}}}
.
\]

The strategy~\(\sigma\Ii\) is optimal for~\(\tuple{u\Ii, \delta\Ii}\) with respect to~\(m\Ii\) and~\(\mu\Ii\).
\end{proposition}

\begin{proof}
The expected payoff at time~\(t\) depends on the model state~\(z\Ii\Tt\) and the action~\(a\Ii\Tt\).
Since the action has no impact on the model state, myopic optimization is sufficient to guarantee global optimality.
\end{proof}

\begin{theorem}
Let the pair~\(\tuple{\mu, \sigma}\) be \aac{xeee} for the game~\(\tuple{u, \delta}\) with mockups~\(m\).
Let~\(\pi\) be the stationary distribution of the Markov chain over the model states~\(z\) induced by~\(\sigma\) and~\(m\).

The pair~\(\tuple{\pi, \sigma}\) is a correlated equilibrium for the one-shot game described by~\(u\).
\end{theorem}

In particular, when all the agents use a memoryless model, the pair~\(\tuple{\pi, \sigma}\) is a Nash equilibrium for~\(u\).

The careful choice of definitions in the previous sections makes the proof of this theorem straightforward.
The key insight of the proof is to interpret the model state~\(z\Ii\) as the type of agent~\(i\).

\begin{proof}
Fix an agent~\(i\).
Pick a state~\(z\Ii \in \cZ\Ii\) and a signal~\(s\Ii\nxt = a\mI \in \cS\Ii\).
By definition of \aac{xeee}, the predictor~\(\mu\Ii\) is consistent with~\(m\) and~\(\sigma\).
The consistency condition for state~\(z\Ii\) and signal~\(a\mI\) can be rewritten as follows:
\begin{align*}
\mu\Ii \of{z\Ii} \elmt{a\mI}
&= \probacond[\pi]{a\mI}{z\Ii} \\
&= \sum\idxin{z\mI}{\cZ\mI} \probacond[\pi]{a\mI}{z\Ii, z\mI} \probacond[\pi]{z\mI}{z\Ii} \\
&= \sum\idxin{z\mI}{\cZ\mI} \probacond[\pi]{a\mI}{z\mI} \probacond[\pi]{z\mI}{z\Ii} \\
&= \sum\idxin{z\mI}{\cZ\mI} \sigma\mI \of{z\mI} \elmt{a\mI} \probacond[\pi]{z\mI}{z\Ii} \\
&= \expectcond[\pi]{\sigma\mI \of{z\mI} \elmt{a\mI}}{z\Ii} \\
&= \expectcond[Z \drawn \pi]{\sigma\mI \of{Z\mI} \elmt{a\mI}}{Z\Ii = z\Ii}
,
\end{align*}
where~\(\sigma\mI \of{z\mI} \elmt{a\mI}\) denotes~\(\prod\idxin{j}{-i} \sigma\Ij \of{z\Ij} \elmt{a\Ij}\).
This equality holds for any \(a\mI\), therefore, ~\(\mu\Ii \of{z\Ii} = \expectcond[Z \drawn \pi]{\sigma\mI \of{Z\mI}}{Z\Ii = z\Ii}\).

Pick an action~\(a\Ii' \in \cA\Ii\).
By definition of \aac{xeee}, the strategy~\(\sigma\Ii\) is optimal with respect to~\(\mu\Ii\).
Substituting the expression for~\(\mu\Ii \of{z\Ii}\) in the optimality condition of~\cref{res:myopic_optimality} gives the following inequality:
\[
\expectcond[Z \drawn \pi]{u\Ii \of{\sigma\Ii \of{z\Ii}, \sigma\mI \of{z\mI}}}{Z\Ii = z\Ii}
\ge
\expectcond[Z \drawn \pi]{u\Ii \of{a\Ii', \sigma\mI \of{z\mI}}}{Z\Ii = z\Ii}
.
\]
Interpreting~\(z\Ii\) as the type of agent~\(i\) in~\cref{res:characterization_CE} guarantees that the pair~\(\tuple{\pi, \sigma}\) is a correlated equilibrium for~\(u\).
\end{proof}

The following section illustrates this result.

\subsection{An Example}

In the hawk-dove game, two agents compete for a prize of value~\(6\).
The actions of each agent are to be aggressive or passive.
In a biological analogy, the aggressive action is called hawk and the passive action is called dove.
If only one agent is aggressive, this agent gets the prize to itself.
If both agents are aggressive, a fight ensues and both agents are hurt.
Finally, if both agents are passive, they split the prize equally.
This story is encoded in the following normal-form game:
\begin{equation*}
\punctuategame{
\begin{game}{2}{2}
        \> \(\rlH\)   \> \(\rlD\)  \\
\(\rH\) \> \(-1, -1\) \> \(6, 0\) \\
\(\rD\) \> \(0, 6\)   \> \(3, 3\)
\end{game}}
{.}\bigskip
\end{equation*}
The actions of agent~\(1\) are represented by the uppercase letters~\(\rH\) and~\(\rD\).
Those of agent~\(2\) by their lowercase counterparts~\(\rlH\) and~\(\rlD\).
Recall that the actions of one agent are the signals of the other.
Using uppercase and lowercase helps in avoiding confusion.

The analysis of the best-response correspondences yields \cref{fig:hawk-dove_best_response}.
The hawk-dove game has two pure Nash equilibria~\(\tuple{\rH, \rlD}\) and~\(\tuple{\rD, \rlH}\), and one mixed Nash equilibrium where the agents play~\(\frac{3}{4} \rH + \frac{1}{4} \rD\) and~\(\frac{3}{4} \rlH + \frac{1}{4} \rlD\) respectively.

\begin{figure}[htp]
\inputtikz{hawk-dove_best_response}
\centering
\ccaption[Best responses and Nash equilibria for hawk-dove.]{
Agent~\(1\) plays~\(\rH\) with probability~\(p\one\).
Agent~\(2\) plays~\(\rlH\) with probability~\(p\two\).
The solid line is agent~\(1\)'s best response.
The dashed line is agent~\(2\)'s best response.
The filled circles indicate the Nash equilibria.
}
\label{fig:hawk-dove_best_response}
\end{figure}

Let us construct \aac{xeee} implementing a correlated-equilibrium distribution.
The correlated-equilibrium distribution chosen is the average of the two pure Nash equilibria~\(\alpha = \frac{1}{2} \tuple{\rH, \rlD} + \frac{1}{2} \tuple{\rD, \rlH}\).
The set of correlated-equilibrium distributions is a non-empty convex set containing all the Nash equilibria.
Therefore, \(\alpha\) is a correlated-equilibrium distribution, even though it is not a Nash equilibrium.
Given the symmetric nature of the game, we chose to implement a symmetric equilibrium, meaning that the strategies of the two agents are identical.
As a consequence, their predictors are also identical.
Both agents use the previously-mentioned depth-\(2\) model.

Let us describe what the solution looks like from agent~\(1\)'s perspective.
Agent~\(1\)'s state is~\(z1 = \tuple{a\two\prv, a\two}\), where~\(a\two\) is the latest observed action of agent~\(2\) and~\(a\two\prv\) the one before that.
If agent~\(1\) sees that agent~\(2\) alternates its actions, it supposes that agent~\(2\) acts according to the plan and that this alternation will continue.
If agent~\(2\) uses the same action twice in a row, agent~\(1\) is unsure about agent~\(2\)'s behavior.
According to these predictions, agent~\(1\) builds optimal or approximately optimal strategies.

Let us now fill in the details.
We provide three variations associated with eventual consistency, consistency, and a new concept called approximate consistency.

\subsubsection{Eventual Consistency}
The first variation is closest to the story previously described.
The agents use the following predictors associated with their depth-\(2\) models:
\[
\begin{aligned}
\mu\one \of{\rlD, \rlH} & = \rlD, \\
\mu\one \of{\rlH, \rlD} & = \rlH, \\
\mu\one \of{\rlH, \rlH} & = \frac{3}{4} \rlH + \frac{1}{4} \rlD, \\
\mu\one \of{\rlD, \rlD} & = \frac{3}{4} \rlH + \frac{1}{4} \rlD, \\
\end{aligned}
\qquad\text{and}\qquad
\begin{aligned}
\mu\two \of{\rD, \rH} & = \rD, \\
\mu\two \of{\rH, \rD} & = \rH, \\
\mu\two \of{\rH, \rH} & = \frac{3}{4} \rH + \frac{1}{4} \rD, \\
\mu\two \of{\rD, \rD} & = \frac{3}{4} \rH + \frac{1}{4} \rD. \\
\end{aligned}
\]
A pair of associated optimal strategies is
\[
\begin{aligned}
\sigma\one \of{\rlD, \rlH} & = \rH, \\
\sigma\one \of{\rlH, \rlD} & = \rD, \\
\sigma\one \of{\rlH, \rlH} & = \frac{1}{2} \rH + \frac{1}{2} \rD, \\
\sigma\one \of{\rlD, \rlD} & = \frac{1}{2} \rH + \frac{1}{2} \rD, \\
\end{aligned}
\qquad\text{and}\qquad
\begin{aligned}
\sigma\two \of{\rD, \rH} & = \rlH, \\
\sigma\two \of{\rH, \rD} & = \rlD, \\
\sigma\two \of{\rH, \rH} & = \frac{1}{2} \rlH + \frac{1}{2} \rlD, \\
\sigma\two \of{\rD, \rD} & = \frac{1}{2} \rlH + \frac{1}{2} \rlD. \\
\end{aligned}
\]
These strategies induce a Markov chain over the state space~\(\cZ\one \times \cZ\two = \cA\two\pow{2} \times \cA\one\pow{2}\).
By computing the transition matrix, one verifies that this Markov chain is unichain and periodic with period two.
Its communication class is~\(\set{\tuple{\rlH, \rlD, \rD, \rH}, \tuple{\rlD, \rlH, \rH, \rD}}\).
In the limit, the chain alternates between these two states and the following relations hold:
\begin{equation}
\label{eq:hawk-dove_unseen_states}
\begin{aligned}
&\limfty{t} \probaof[\rbN, \sigma]{z\one\Tt = \tuple{\rlD, \rlD}} = 0, \\
&\limfty{t} \probaof[\rbN, \sigma]{z\one\Tt = \tuple{\rlH, \rlH}} = 0,
\end{aligned}
\qquad\text{and}\qquad
\begin{aligned}
&\limfty{t} \probaof[\rbN, \sigma]{z\two\Tt = \tuple{\rD, \rD}} = 0, \\
&\limfty{t} \probaof[\rbN, \sigma]{z\two\Tt = \tuple{\rH, \rH}} = 0.
\end{aligned}
\end{equation}
In the limit, \(14\) out of the~\(16\) states do not appear.
In particular, any state in which an agent used the same action twice in a row is transient.
Therefore,
\begin{equation}
\label{eq:hawk-dove_alternating_states}
\begin{aligned}
&\limfty{t} \probacond[\rbN, \sigma]{s\one\Tp = \rlD}{z\one\Tt = \tuple{\rlD, \rlH}} = 1, \\
&\limfty{t} \probacond[\rbN, \sigma]{s\one\Tp = \rlH}{z\one\Tt = \tuple{\rlH, \rlD}} = 1, \\
&\limfty{t} \probacond[\rbN, \sigma]{s\two\Tp = \rD}{z\two\Tt = \tuple{\rD, \rH}} = 1, \\
&\limfty{t} \probacond[\rbN, \sigma]{s\two\Tp = \rH}{z\two\Tt = \tuple{\rH, \rD}} = 1.
\end{aligned}
\end{equation}
\Cref{eq:hawk-dove_unseen_states} guarantees that states~\(\tuple{\rlH, \rlH}\) and~\(\tuple{\rlD, \rlD}\) vanish.
Therefore, the values of~\(\mu\one \of{\rlH, \rlH}\) and~\(\mu\one \of{\rlD, \rlD}\) are arbitrary.
The values of~\(\mu\one \of{\rlD, \rlH}\) and~\(\mu\one \of{\rlH, \rlD}\) match the values observed in~\cref{eq:hawk-dove_alternating_states}.
Therefore, the predictors are eventually consistent.

Let us now look at the optimality condition.
Eventual consistency allows for the predictors to take arbitrary values on transient states~\(\tuple{\rlH,\rlH}\) and~\(\tuple{\rlD,\rlD}\).
Therefore, we chose values helping with the optimality condition.
The values chosen correspond to the mixed Nash equilibrium in which the agents are indifferent between their two actions.
Agent~\(1\) responds optimally in each of the four states~\(z\one\).

Therefore, we have constructed an exact \ac{xeee} with the notion of eventual consistency which implements the desired correlated-equilibrium distribution.

\subsubsection{Consistency}

Let us now go from eventual consistency to consistency.
The agents use strategies
\[
\begin{aligned}
\sigma\one \of{\rlD, \rlH} & = 0.999 \, \rH + 0.001 \, \rD, \\
\sigma\one \of{\rlH, \rlD} & = 0.999 \, \rD + 0.001 \, \rH, \\
\sigma\one \of{\rlH, \rlH} & = \frac{1}{2} \rH + \frac{1}{2} \rD, \\
\sigma\one \of{\rlD, \rlD} & = \frac{1}{2} \rH + \frac{1}{2} \rD, \\
\end{aligned}
\qquad\text{and}\qquad
\begin{aligned}
\sigma\two \of{\rD, \rH} & = 0.999 \, \rlH + 0.001 \, \rlD, \\
\sigma\two \of{\rH, \rD} & = 0.999 \, \rlD + 0.001 \, \rlH, \\
\sigma\two \of{\rH, \rH} & = \frac{1}{2} \rlH + \frac{1}{2} \rlD, \\
\sigma\two \of{\rD, \rD} & = \frac{1}{2} \rlH + \frac{1}{2} \rlD. \\
\end{aligned}
\]
The induced Markov chain over~\(\cZ\one \times \cZ\two\) is irreducible and aperiodic.
No state is transient.
Therefore, predictors have to be defined for all states.
The consistent predictors are
\[
\begin{aligned}
\mu\one \of{\rlD, \rlH} & = 0.996 \, \rlD + 0.004 \, \rlH, \\
\mu\one \of{\rlH, \rlD} & = 0.996 \, \rlH + 0.004 \, \rlD, \\
\mu\one \of{\rlH, \rlH} & = 0.5 \, \rlH + 0.5 \, \rlD, \\
\mu\one \of{\rlD, \rlD} & = 0.5 \, \rlH + 0.5 \, \rlD, \\
\end{aligned}
\qquad\text{and}\qquad
\begin{aligned}
\mu\two \of{\rD, \rH} & = 0.996 \, \rD + 0.004 \, \rH, \\
\mu\two \of{\rH, \rD} & = 0.996 \, \rH + 0.004 \, \rD, \\
\mu\two \of{\rH, \rH} & = 0.5 \, \rH + 0.5 \, \rD, \\
\mu\two \of{\rD, \rD} & = 0.5 \, \rH + 0.5 \, \rD. \\
\end{aligned}
\]
The probabilities reported as~\(0.5\) are not exactly~\(\frac{1}{2}\).
These probabilities are biased towards the symbol just observed twice with a bias of the order of~\(10\pow{-12}\).

In this setting, consistency is immediate, by definition of the predictors.
Optimality is slightly trickier.
Recall that in \aac{xeee} for a perfect-monitoring repeated game, the optimality condition translates to myopic optimality.
Therefore, \(\sigma\one\)'s optimality with respect to~\(\mu\one\) and~\(m\one\) is equivalent to the following condition.
For all~\(z\one \in \cZ\one\), agent~\(1\)'s mixed action~\(\sigma\one \of{z\one}\) is a best response to agent~\(2\)'s mixed action~\(\mu\one \of{z\one}\).
This is not the case for the states~\(\tuple{\rlD, \rlD}\) and~\(\tuple{\rlH, \rlH}\).
As seen in \cref{fig:hawk-dove_best_response}, agent~\(1\)'s sole best-response to~\(0.5 \, \rlH + 0.5 \, \rlD\) is~\(\rH\).
However, \(\sigma\one\) is approximately optimal with respect to~\(\mu\one\) for discount factors~\(\delta\one\) close enough to one.
Most of the time is spent in states~\(\tuple{\rlD, \rlH}\) and~\(\tuple{\rlH, \rlD}\) for which~\(\sigma\one\) is optimal.
By taking~\(\delta\one\) close enough to one, the effect of acting non-optimally in the other two states becomes negligible.

\Cref{res:myopic_optimality} proves that myopic optimality is sufficient to get optimality.
This example illustrates that approximate optimality does not require approximate myopic optimality.
Strategies can perform poorly on vanishing states.

The resulting equilibrium is an approximate \ac{xeee}.
The associated distribution over actions
\[
\begin{matrix}
0.004 \, \tuple{\rH, \rlH} & 0.496 \, \tuple{\rH, \rlD} \\
0.496 \, \tuple{\rD, \rlH} & 0.004 \, \tuple{\rD, \rlD}
\end{matrix},
\]
is an approximation of the desired correlated-equilibrium distribution.

\subsubsection{Approximate Consistency}

To conclude this example, let us analyze what happens when approximately consistent predictors are used.
The agents use the same smoothed strategies
\[
\begin{aligned}
\sigma\one \of{\rlD, \rlH} & = 0.999 \, \rH + 0.001 \, \rD, \\
\sigma\one \of{\rlH, \rlD} & = 0.999 \, \rD + 0.001 \, \rH, \\
\sigma\one \of{\rlH, \rlH} & = \frac{1}{2} \rH + \frac{1}{2} \rD, \\
\sigma\one \of{\rlD, \rlD} & = \frac{1}{2} \rH + \frac{1}{2} \rD, \\
\end{aligned}
\qquad\text{and}\qquad
\begin{aligned}
\sigma\two \of{\rD, \rH} & = 0.999 \, \rlH + 0.001 \, \rlD, \\
\sigma\two \of{\rH, \rD} & = 0.999 \, \rlD + 0.001 \, \rlH, \\
\sigma\two \of{\rH, \rH} & = \frac{1}{2} \rlH + \frac{1}{2} \rlD, \\
\sigma\two \of{\rD, \rD} & = \frac{1}{2} \rlH + \frac{1}{2} \rlD, \\
\end{aligned}
\]
and these predictors
\[
\begin{aligned}
\mu\one \of{\rlD, \rlH} & = 0.999 \, \rlD + 0.001 \, \rlH, \\
\mu\one \of{\rlH, \rlD} & = 0.999 \, \rlH + 0.001 \, \rlD, \\
\mu\one \of{\rlH, \rlH} & = \frac{1}{2} \rlH + \frac{1}{2} \rlD, \\
\mu\one \of{\rlD, \rlD} & = \frac{1}{2} \rlH + \frac{1}{2} \rlD, \\
\end{aligned}
\qquad\text{and}\qquad
\begin{aligned}
\mu\two \of{\rD, \rH} & = 0.999 \, \rD + 0.001 \, \rH, \\
\mu\two \of{\rH, \rD} & = 0.999 \, \rH + 0.001 \, \rD, \\
\mu\two \of{\rH, \rH} & = \frac{1}{2} \rH + \frac{1}{2} \rD, \\
\mu\two \of{\rD, \rD} & = \frac{1}{2} \rH + \frac{1}{2} \rD. \\
\end{aligned}
\]
These predictors are necessarily inconsistent.
However, they are approximately consistent, meaning close to the consistent ones.
Since the strategies are approximately optimal for the consistent predictors, \cref{res:markov_continuity} guarantees they are also approximately optimal for the approximately consistent predictors.
Therefore, the pair~\(\tuple{\sigma, \mu}\) forms an approximate~\ac{eee} in the sense of \cref{def:eee}.

This example explains why we did not formally define the notion of approximate consistency.
\Cref{res:markov_continuity} guarantees that we can trade approximate consistency for approximate optimality.
For the sake of clarity, it is easier to only have approximation in one place and we chose to have approximate optimality.

\subsection{Average of Nash Equilibria}

The example of the previous section easily extends to general finite games and yields a large set of correlated-equilibrium distributions.
Most of the work for the proof has already been done in the example.

\begin{theorem}
\label{res:xeee_and_ce}
Let~\(u \from \cA \to \bR\pow{\CI}\) describe a one-shot game.
Let~\(\seqfromto{a}{\tm}{l}{1}{k}\) be~\(k\), non-necessarily distinct, pure Nash equilibria of~\(u\).
Denote by~\(\alpha \in \distribover{\cA}\) the average of these Nash equilibria
\[
\alpha = \frac{1}{k} \sum\idxfromto{l}{1}{k} a\tm{l},
\]
which is a correlated-equilibrium distribution.

For large enough discount factors~\(\seqin{\delta}{\ag}{i}{\cI}\), \(~\alpha\) is implementable by an approximate~\ac{xeee}, in which each agent uses a depth-\(k\) eventually consistent model.
\end{theorem}

\begin{proof}
Let~\(i \in \cI\) be an agent.
Define its depth-\(k\) predictor as follows:
\[
\begin{aligned}
\mu\Ii \of {a\mI\tm{1}, a\mI\tm{2}, \cdots, a\mI\tm{k - 1}, a\mI\tm{k}} & = a\mI\tm{1}, \\
\mu\Ii \of {a\mI\tm{2}, a\mI\tm{3}, \cdots, a\mI\tm{k}, a\mI\tm{1}} & = a\mI\tm{2}, \\
&\vdotswithin{=} \\
\mu\Ii \of {a\mI\tm{k}, a\mI\tm{1}, \cdots, a\mI\tm{k - 2}, a\mI\tm{k - 1}} & = a\mI\tm{k},
\end{aligned}
\]
and for all the other states~\(z\Ii\),
\[
\mu\Ii \of {z\Ii} = \frac{1}{k} \sum\idxfromto{l}{1}{k} a\mI\tm{k}.
\]
As opposed to the example, a mixed Nash equilibrium does not always exist.
This reduced flexibility in defining the predictor on vanishing states, explains why only approximate~\acp{xeee} are guaranteed.

Define agent~\(i\)'s strategy as follows:
\[
\begin{aligned}
\sigma\Ii \of {a\mI\tm{1}, a\mI\tm{2}, \cdots, a\mI\tm{k - 1}, a\mI\tm{k}} &= a\Ii\tm{1}, \\
\sigma\Ii \of {a\mI\tm{2}, a\mI\tm{3}, \cdots, a\mI\tm{k}, a\mI\tm{1}} &= a\Ii\tm{2}, \\
&\vdotswithin{=} \\
\sigma\Ii \of {a\mI\tm{k}, a\mI\tm{1}, \cdots, a\mI\tm{k - 2}, a\mI\tm{k - 1}} &= a\Ii\tm{k},
\end{aligned}
\]
and for all the other states~\(z\Ii\),
\[
\sigma\Ii \of {z\Ii} = \frac{1}{k} \sum\idxfromto{l}{1}{k} a\Ii\tm{k}.
\]
For a vanishing state~\(z\Ii\), the definition of~\(\mu\Ii \of {z\Ii}\) could have been anything.
The definition of~\(\sigma\Ii \of {z\Ii}\) requires that each action appearing in one of the~\(k\) pure Nash equilibria appear with positive probability.

The induced Markov chain is unichain and periodic with period~\(k\).
Its communication class has~\(k\) states corresponding to each of the~\(k\) Nash equilibria.
In the limit, the chain cycles through these~\(k\) states in the order imposed by the labeling of the equilibria.
The eventual consistency of the predictors is proven as in the example.

As previously mentioned, it is not always possible to guarantee optimality of~\(\sigma\Ii\) with respect to~\(\mu\Ii\).
However, \(\sigma\Ii\) is optimal for all the states visited in the limit.
The lack of optimality is only for vanishing states.
Therefore, a large enough discount factor guarantees approximate optimality.
\end{proof}

\begin{corollary}
Let~\(u \from \cA \to \bR\pow{\CI}\) describe a one-shot game.
Let~\(\seqfromto{a}{\tm}{l}{1}{k}\) be~\(k\), non-necessarily distinct, pure Nash equilibria of~\(u\).
Denote by~\(\alpha \in \distribover{\cA}\) the average of these Nash equilibria
\[
\alpha = \frac{1}{k} \sum\idxfromto{l}{1}{k} a\tm{l},
\]
which is a correlated-equilibrium distribution.

For large enough discount factors~\(\seqin{\delta}{\ag}{i}{\cI}\), \(~\alpha\) can be approximated by a correlated-equilibrium distribution induced by an approximate~\ac{xeee}, in which each agent uses a depth-\(k\) consistent model.
\end{corollary}

\begin{proof}
Let~\(i \in \cI\) be an agent and~\(\epsilon > 0\).
Denote by~\(\omega\Ii\) the uniform distribution over~\(\cA\Ii\).
Define agent~\(i\)'s strategy as follows:
\[
\begin{aligned}
\sigma\Ii \of {a\mI\tm{1}, a\mI\tm{2}, \cdots, a\mI\tm{k - 1}, a\mI\tm{k}} &= \onem{\epsilon} a\Ii\tm{1} + \epsilon\omega\Ii, \\
\sigma\Ii \of {a\mI\tm{2}, a\mI\tm{3}, \cdots, a\mI\tm{k}, a\mI\tm{1}} &= \onem{\epsilon}a\Ii\tm{2} + \epsilon\omega\Ii, \\
&\vdotswithin{=} \\
\sigma\Ii \of {a\mI\tm{k}, a\mI\tm{1}, \cdots, a\mI\tm{k - 2}, a\mI\tm{k - 1}} &= \onem{\epsilon}a\Ii\tm{k} + \epsilon\omega\Ii,
\end{aligned}
\]
and for all the other states~\(z\Ii\),
\[
\sigma\Ii \of {z\Ii} = \onem{\epsilon} \frac{1}{k} \sum\idxfromto{l}{1}{k} a\Ii\tm{k} + \epsilon\omega\Ii.
\]

The joint strategy~\(\sigma = \agstuple{\sigma}\) induces an irreducible aperiodic Markov chain.
Define~\(\mu = \agstuple{\mu}\) as the associated consistent predictors.

Using the same proofs as in the example, one shows that, for~\(\epsilon\) small enough, the pair~\(\tuple{\sigma, \mu}\) forms an approximate \ac{xeee}.
Furthermore, this construction induces an approximation of the correlated-equilibrium distribution~\(\alpha\).
\end{proof}

\section{Learning Empirical-evidence Equilibria}
\label{sec:learning_empirical_evidence_equilibria}

\subsection{A Learning Rule}

The fixed points of~\(\GGe\) are~\(\epsilonm\) \acp{eee}.
A natural approach to try and learn an~\(\epsilonm\) \ac{eee} is to use an adaptive rule that converges to fixed points.
Consider the following adaptive rule:
\begin{equation}
\label{eq:update_rule}
\mu\Tp = \mu\Tt + \alpha\Tt \parentheses*{\GGe \of{\mu\Tt} - \mu\Tt},
\end{equation}
where~\(\alpha\Tt\) is a step size.
The long-run behavior of \cref{eq:update_rule} is related to properties of the following differential equation:
\begin{equation}
\label{eq:differential_equation}
\dot{\mu} = \GGe \of{\mu} - \mu.
\end{equation}
In particular, Benaïm showed that the limit set of \cref{eq:update_rule} is a connected set internally chain recurrent for the flow induced by~\(\GGe - \mathrm{Id}\), where~\(\mathrm{Id}\) is the identity function~\cite{benaim:1996}.
The following, easily verified, proposition tells us that, if \cref{eq:update_rule} converges, it might yield an~\(\epsilonm\)~\ac{eee}.

\begin{proposition}
The fixed points of~\(\GGe\) are connected sets internally chain recurrent for the flow induced by~\(\GGe - \mathrm{Id}\).
\end{proposition}

Note, however, that these fixed points might not be the only connected sets internally chain recurrent for this flow.

The existence of a Lyapunov function for~\cref{eq:differential_equation} guarantees that there is a unique connected sets internally chain recurrent for the flow induced by~\(\GGe - \mathrm{Id}\) which is the equilibrium point.
Therefore, if there exists such a Lyapunov function for the continuous system, we can conclude that \cref{eq:update_rule} converges to \aac{eee}.

\subsection{Simulation Results}
This learning rule was successfully used on a simplified market example.
Two agents can hold a quantity of a single asset between~\(0\) and~\(4\), \(\cX = \set{0,1,2,3,4}\).
At each time step, each agent can sell one asset, buy one asset, or hold its position, \(\cA = \set{\text{Sell}, \text{Hold}, \text{Buy}}\).
The assets can be traded at a low price or at a high price, \(\cS=\set{\text{Low}, \text{High}}\).
Nature exogenously determines the market trend as a bull market or a bear market, \(\cW=\cS \times \set{\text{Bear}, \text{Bull}}\).
The price is impacted by the past price, the market trend, and the orders placed by the two agents.
A high price in the past, buying orders, or a bull market increase the chances of seeing a high price in the future.
The agents receive the price at each time step but are not aware of the price dynamic.
In this model, they are not even aware of the existence of the market trend.
The two agents use a discount factor~\(\delta=0.95\).

Agent~1 starts with the idea that the price will be high with probability 1.
Agent~2 starts with the idea that the price will be low with probability 1.
Each agent is trying to learn a depth-\(0\) model of the price.
Two versions of~\cref{eq:update_rule} were simulated.
The first one used~\cref{eq:update_rule} directly with a fixed step size of~\(\alpha\Tt = 0.1\).
The stationary distribution~\(\pi\strat{\sigma}\) was computed at each time step to obtain the true value of~\(\GGe \of{\mu\Tt}\).
The algorithm is presented in~\cref{alg:theoretical_predictor_learning}.
The results of the simulations using the theoretical predictor are presented in~\cref{fig:simulation_theoretical}.
Since the price is a public signal, after a transient phase due to the step size, the predictions of both agents agree.
The prediction converges to probability of seeing a high price of~\(0.431\).
The two agents use the same strategy that is the optimal response for that prediction of the price.
When the price is high sell.
When the price is low, sell when having four units, hold when having three units, and buy otherwise.
The learning rule has indeed converged to an \ac{eee}.

In the second simulated version of~\cref{eq:update_rule}, the stationary distribution is only estimated by playing~\(100\) rounds of the game at each time step.
Because of the variance induced by this sampling process, the step size was taken to be diminishing, \(\alpha\Tt = \parentheses*{\frac{1}{t}}\pow{\frac{3}{4}}\).
The estimated predictors obtained in that case are denoted by~\(\hat{\mu}\Ii\Tt\).
The algorithm is presented in~\cref{alg:empirical_predictor_learning}.
The results of the simulation using the empirical predictors are presented in~\cref{fig:simulation_empirical}.
Estimating, instead of using the true probability, induces some variations.
The learning rule does not converge, but oscillates around the \ac{eee} reached by the theoretical predictor.

\newcommand{\figureswidth}{320pt}
\newcommand{\figuresheight}{170pt}

\begin{figure}[htp]
  \centering
  \newcommand{\ylabelsymbol}{\(\mu\Tt\Ii\elmt{\mathrm{High}}\)}%
  \newcommand{\tabledata}{figures/simulation_theoretical.dat}%
  \inputtikz{simulation}
  \caption{Simulation results of two agents learning a depth-0 model of the price for the market example with the theoretical predictor.}
  \label{fig:simulation_theoretical}
\end{figure}

\begin{figure}[htp]
  \centering
  \newcommand{\ylabelsymbol}{\(\hat{\mu}\Tt\Ii\elmt{\mathrm{High}}\)}%
  \newcommand{\tabledata}{figures/simulation_empirical.dat}%
  \inputtikz{simulation}
  \caption{Simulation results of two agents learning a depth-0 model of the price for the market example with an empirical predictor.}
  \label{fig:simulation_empirical}
\end{figure}

\begin{algorithm}[pht]
  \caption{Learning with Theoretical Predictor}
  \label{alg:theoretical_predictor_learning}
  \begin{algorithmic}
    \Procedure{Theoretical Predictor Learning}{$\epsilon$}
      \State \(\mu\one \gets 1\)
      \State \(\mu\two \gets 0\)
      \For{\(t \in \interval{1}{100}\)}
        \State \(\sigma\one \gets\) an optimal strategy for~\(\mu\one\)
        \State \(\sigma\two \gets\) an optimal strategy for~\(\mu\two\)
        \State \(T\param{\sigma} \gets\) the transition matrix for the Markov chain induced by~\(\sigma\one\) and~\(\sigma\two\)
        \State \(\pi\param{\sigma} \gets\) the eigenvector associated with the eigenvalue~\(1\) for~\(T\param{\sigma}\)
        \State \(\hat{\mu} \gets \probaof[\pi\strat{\sigma}]{\mathrm{High}}\)
        \State \(\mu\one \gets \mu\one + 0.1 \grpparen{\hat{\mu} - \mu\one}\)
        \State \(\mu\two \gets \mu\two + 0.1 \grpparen{\hat{\mu} - \mu\two}\)
      \EndFor
    \EndProcedure
  \end{algorithmic}
\end{algorithm}

\begin{algorithm}[pht]
  \caption{Learning with Empirical Predictor}
  \label{alg:empirical_predictor_learning}
  \begin{algorithmic}
    \Procedure{Empirical Predictor Learning}{$\epsilon$}
      \State \(\mu\one \gets 1\)
      \State \(\mu\two \gets 0\)
      \For{\(t \in \interval{1}{100}\)}
        \State \(\sigma\one \gets\) an optimal strategy for~\(\mu\one\)
        \State \(\sigma\two \gets\) an optimal strategy for~\(\mu\two\)
        \State \(h \gets 0\)
        \For{\(\tau \in \interval{1}{100}\)}
          \State agent~\(1\) places an order according to~\(\sigma\one\)
          \State agent~\(2\) places an order according to~\(\sigma\two\)
          \If{the observed price is High}
            \State \(h \gets h+1\)
          \EndIf
        \EndFor
        \State \(\hat{\mu} \gets \frac{h}{100}\)
        \State \(\mu\one \gets \mu\one + \grpparen{\frac{1}{t}}\pow{\frac{3}{4}} \grpparen{\hat{\mu} - \mu\one}\)
        \State \(\mu\two \gets \mu\two + \grpparen{\frac{1}{t}}\pow{\frac{3}{4}} \grpparen{\hat{\mu} - \mu\two}\)
      \EndFor
    \EndProcedure
  \end{algorithmic}
\end{algorithm}

\subsection{Effect of the Finite Observation Window}

In the empirical simulation, approximate predictors are used.
The function~\(\FM\) computes predictors consistent with given strategies through the stationary distribution of a Markov chain.
The function~\(\FM\Ll\) is obtained when approximate predictors are instead computed from~\(l\) rounds of play.
The following proposition shows that important properties of~\(\FM\) are recovered by~\(\FM\Ll\), as~\(l\) goes to infinity.
A proof sketch for this proposition is given to highlight useful tools in this context.
Given a real problem, the proposition can be adapted, using the appropriate textbook material on stochastic approximation and Lyapunov stability of perturbed systems.

\begin{assumption}
\label{ass:non_stochastic}
When computing~\(\FM\Ll\), at the beginning of the~\(l\) rounds of play, the state of Nature~\(w\), the states of the agents~\(x\), the state of the models~\(z\), and the seed to the pseudorandom number generators are reset.
\end{assumption}

\begin{proposition}
Under \cref{ass:non_stochastic}, if there exists a Lyapunov function for the following continuous-time system:
\begin{equation}
\label{eq:continuous_ideal}
\dot\mu = \GGe \of{\mu} - \mu,
\end{equation}
then for~\(l\) large enough, the following discrete-time system, using approximate predictors, converges to \aac{eee}:
\begin{equation}
\label{eq:discrete_finite-observation}
\muhat\Tp = \muhat\Tt + \alpha\Tt \grpparen{\GGe\Ll \of{\muhat\Tt} - \muhat\Tt},
\end{equation}
where \(\GGe\Ll = \FM\Ll \compo \FOe\) and~\(\seqin{\alpha}{\tm}{t}{\bN}\) is a non summable but square summable sequence of positive numbers.
\end{proposition}

The proof starts with the use of dynamical system
\begin{equation}
\label{eq:continuous_finite-observation}
\dot\muhat = \GGe\Ll \of{\muhat} - \muhat,
\end{equation}
and Lyapunov stability of perturbed systems such as~\cite[\citelemma{9.1} or \citelemma{9.2}]{khalil:2002}.
For a large enough~\(l\), dependent on the chosen lemma, \cref{eq:continuous_finite-observation} constitutes a perturbed version of \cref{eq:continuous_ideal}.
Therefore, for~\(l\) large enough, a Lyapunov function for~\cref{eq:continuous_ideal} is also a Lyapunov function for \cref{eq:continuous_finite-observation}.
The existence of a Lyapunov function for \cref{eq:continuous_finite-observation} implies that the only connected set internally chain recurrent for the flow induced by \(\GGe\Ll -\mathrm{Id}\) is the singleton containing the equilibrium point.
\Cref{ass:non_stochastic} allows the application of deterministic stochastic-approximation results.
In particular, \cite[\citethm{1.2}]{benaim:1996} guarantees that the limit set of the sequence \(\seqin{\muhat}{\tm}{t}{\bN}\), solution to \cref{eq:discrete_finite-observation}, is a connected set internally chain recurrent for the flow induced by \(\GGe\Ll\).
Therefore, \(\seqin{\muhat}{\tm}{t}{\bN}\) converges to \aac{eee}.
